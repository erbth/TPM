% !TeX spellcheck = en_US
\documentclass[a4paper]{article}
\usepackage[left=2cm, right=2cm, top=2cm, bottom=2cm]{geometry}
\usepackage{lmodern}
\usepackage[T1]{fontenc}
\usepackage[english]{babel}
\usepackage{tabularx}
\usepackage{xcolor}
\usepackage{dirtree}

\newcommand{\file}[1]{\texttt{#1}}
\newcommand{\program}[1]{\textbf{#1}}
\newcommand{\variable}[1]{'\texttt{#1}'}

\newcommand{\green}[1]{\textcolor{green}{#1}}

\title{TSClient LEGACY Package Manager}
\author{Thomas Erbesdobler <t.erbesdobler@gmx.de>}

\begin{document}
	\maketitle
	\tableofcontents
	
	\section{Introduction}
	\label{sec:introduction}
	
	This manual should be read in a linear manner. You must expect that each line depends on the previous text.
	
	\section{Packages}
	\label{sec:packages}
	
	A package can take three different ''shapes'': unpacked (during development), packed (also transport shape) and installed (unpacked without meta data). The meta data od a package is stored in a central database located at \file{/var/lib/tpm/status.xml} during installation.Hence, the meta data is not required to be part of the installed shape. It would furthermore be redundant.
	
	\subsection{Description of the three shapes}
	\label{sec:description_of_the_three_shapes}
	
	\paragraph{unpacked:}
	The unpacked shape is the following directory structure:
	\begin{description}
		\item[\file{\dots/desc.xml}] Describes the package by its type, name, version, computer architecture, runtime- and compiletime dependencies, and files. Two different types of packages exist, one for packaging software and one for bundling and deploying config files, like \file{/etc/inittab}. A file may be a config file, which is only installed if the package is newly installed (not upgraded) and it is not already present, or a non-config file or a directory. Directories are special in that they will only be removed during removal of the package if they are not empty. The software shall be responsible to upgrade previous config files during an upgrade of the package. This can be done using a preupgrade script. For an explanation have a look below.
		
		\file{desc.xml} can be created and modified using \program{tpm}, the single tool of TPM.
		
		\item[\file{\dots/destdir/\dots}] The packages directory structure, which is a tree rooted at \file{/}. Therefore, this is copied to \file{/} of the target system when the package is installed. However config files are treated specially, see above.
		
		The name 'destdir' originates in the Make variable \variable{DESTDIR} that can be used to install software that is built with GNU Autotools to a different directory.
		
		\item[\file{configure.sh}] This script is executed after all runtime dependencies of the package were installed, and if \program{tpm} is invoked with \file{/} as target system (e.g. in a chroot-environment or on a real system).
		
		\item[\file{unconfigure.sh}] \dots
	\end{description}

	\paragraph{packed/transport:}
	An uncompressed tar archive of the following structure:
	\begin{description}
		\item[\file{desc.xml}] '\file{desc.xml}'
		
		\item[\file{destdir.tar.gz}] Gzipped tar archive of \file{destdir}. I choose Gzip because with Pigz a fully compatible parallel implementation exists.
		
		\item[\file{configure.sh}] '\file{configure.sh}'
		
		\item[\file{unconfigure.sh}] '\file{unconfigure.sh}'
	\end{description}
	
	\noindent
	The name of the archive must be <package name>\_<architecture>.tpm.tar. Ther version and architecture strings must follow the same scheme as those in \file{desc.xml} described later.
	
	\paragraph{installed:}
	During installation \program{tpm} tests if any non-config file in destdir is present in the runtime system's directory structure. 'runtime system' refers to the target system to which the package is installed. If this is not the case, it extracts the content of \file{destdir.tar.gx} to the runtime system's root directory. Otherwise \program{tpm} signals an error and does not modify the runtime system's state. If the package is newely installed and features config files, \program{tpm} installs all non present ones. Before any modification is made, \program{tpm} inserts the package into its database (also referred to as 'status') and flags it as 'installation in progress'. Finally, this label is removed and the entry possibly updated to contain all files which are currently installed for the package. This procedure makes modifications atomic.
	
	\subsection{Package types}
	\label{sec:package_types}
	
	To types of package exist: 'sw' (abbreviation of 'software', regular packages which contain executables, libraries etc.) and 'conf' (bundle system config files, i.e. those located in \file{/etc}). Packages of type conf differ in how config files are handled. If installed newly, an error will be signalled and no action be performed if any config file is already present. During an upgrade the same thing happens, however if the file was installed by a former version of the package and not modified since then, \program{tpm} replaces it.
	
	\section{The program \program{tpm}}
	\label{sec:the_program_tpm}
	
	TPM features only one tool, which contains the entire functionality: \program{tpm}.
	
	\subsection{Package management commands}
	\label{sec:package_management_commands}
	
	\bgroup
	\def\arraystretch{1.5}
	\begin{tabularx}{\textwidth}{lX}
		\green{\texttt{-{}-install <name1>, \dots}} & Install or upgrade the specified packages \\
		
		\green{\texttt{-{}-reinstall <name1>, \dots}} & Like install but reinstalls the specified packages even if the same version is already installed \\
		
		\green{\texttt{-{}-policy <name>}} & Show the installed and available versions of <name> \\
		
		\green{\texttt{-{}-show-version <name>}} & Print a package's version number or `\texttt{-{}-{}-}' if it is not installed; This operation is intended to automatically query for installed packages. \\
		
		\green{\texttt{-{}-mark-auto <name1>, \dots}} & Mark the specified packages as automatically installed \\
		
		\green{\texttt{-{}-mark-manual <name1>, \dots}} & Mark the specified packages as manually installed \\
		
		\green{\texttt{-{}-remove <name1>, \dots}} & Remove the specified packages and their config files if they were not modified. If other packages depend on a package which is to be removed, these are removed first without asking the user for confirmation. \\
		
		\green{\texttt{-{}-remove-unneeded}} & Remove all packages that were marked as automatically installed and are not required by other packages that are marked as manually installed \\
		
		\green{\texttt{-{}-list-installed}} & List all installed packages \\
		
		\green{\texttt{-{}-show-problems}} & Show all problems with the current installation (i.e. halfly installed packages after an interruption or missing dependencies) \\
		
		\green{\texttt{-{}-recover}} & Recover from a dirty state by deleting all packages that are in a dirty state (always possible due to atomic write operations to status) \\
		
		\texttt{-{}-upgrade} & Upgrade all upgradeable packages \\
		
		\texttt{-{}-install-from-file <name.tpm.tar>} & Install or upgrade from a \texttt{.tpm.tar} file \\
		
		\green{\texttt{-{}-installation-graph <name1>, \dots}} & Print the dependency graph in the dot format; If packages are specified, they are added to the graph. \\
		
		\green{\texttt{-{}-reverse-dependencies <name>}} & List the installed packages that depend on the specified package directly or indirectly. If the package is part of a cycle in the dependency graph, it is not printed even though it depends on one of its dependents and therefore on its own. \\
		
	\end{tabularx}
	\egroup

	\vspace{1em}
	Any of these commands may be combined with the following options: \\
	\bgroup
	\def\arraystretch{1.5}
	\begin{tabularx}{\textwidth}{lX}
		\texttt{-{}-target} & Root of the system which is managed; This can be specified with the environment variable \variable{TPM\_TARGET}, however \texttt{-{}-target} precedes. \\
	\end{tabularx}
	\egroup
	
	\subsubsection{Details on \texttt{-{}-show-problems}}
	
	\texttt{-{}-show-problems} detects and lists the following types of problems:
	
	\noindent
	\begin{itemize}
		\item Packages in a dirty state, which is any state except 'installed'
		\item Packages that are installed multiple times
		\item Packages of which runtime dependencies are not installed or not configured
	\end{itemize}

	\paragraph{Noncritical problems}
	\noindent
	\begin{itemize}
		\item Dependency violations
		\item Packages that are installed but not configured and also not in the state configuring
	\end{itemize}
	
	\subsection{Creating packages}
	\label{sec:creating_packages}
	
	\bgroup
	\def\arraystretch{1.5}
	\begin{tabularx}{\textwidth}{lX}
		\green{\texttt{-{}-create-desc <type>}} & Create \file{desc.xml} with package type and \file{destdir} in the current working directory \\
		
		\green{\texttt{-{}-set-name <name>}} & Set the package's name \\
		
		\green{\texttt{-{}-set-version <X>}} & Set the package's version to X \\
		
		\green{\texttt{-{}-set-arch <Y>}} & Set the package's architecture to Y \\
		
		\green{\texttt{-{}-add-files}} & Add files from destdir \\
		
		\texttt{-{}-guess-rdependencies} & Guess the package's runtime dependencies from destdir; This does not remove dependencies. However, this does not make sense without a solid package base that includes all used libraries. \\
		
		\green{\texttt{-{}-add-dependency <name>}} & Add the runtime dependency <name> \\
		
		\green{\texttt{-{}-remove-dependencies}} & Remove all runtime dependencies \\
		
		\green{\texttt{-{}-show-missing}} & List information that is missing in the package description \\
		
		\green{\texttt{-{}-pack}} & Generate packed shape in \texttt{.tpm.tar}
	\end{tabularx}
	\egroup

	\subsection{Other command line parameters}
	\label{sec:other_command_line_parameters}
	
	\bgroup
	\def\arraystretch{1.5}
	\begin{tabularx}{\textwidth}{lX}
		\green{\texttt{-{}-version}} & Print \program{tpm}'s version \\
		\green{\texttt{-{}-help}} & Print a help text
	\end{tabularx}
	\egroup
	
	\subsection{Environment variables}
	\label{sec:environment_variables}
	
	\bgroup
	\def\arraystretch{1.5}
	\begin{tabularx}{\textwidth}{lX}
		\variable{TPM\_TARGET} & Specifies the runtime system's root directory; May be overwritten by \texttt{-{}-target} \\
		
		\variable{TPM\_PROGRAM\_SHA512SUM} & The \program{sha512sum} compatible program to use for generating the hash sums of files \\

		\variable{TPM\_PROGRAM\_TAR} & The Tar compatible archiver to use for packing/unpacking the packages \\

		\variable{TPM\_PROGRAM\_GZIP} & The Gzip compatible compression program to use for compressing packages \\
	\end{tabularx}
	\egroup
	
	\section{The program \program{tpmdb}}
	\label{sec:the_program_tpmdb}
	
	\program{tpmdb} can be used to generate a database that contains meta informations of packages. This database can be searched for files and to which package they belong.
	
	\subsection{Commands}
	\bgroup
	\def\arraystretch{1.5}
	\begin{tabularx}{\textwidth}{lX}
		\texttt{-{}-create-from-directory <directory>} & Recursively traverses the specified directory and adds each TPM package it encounters to the database. \\
		
		\texttt{-{}-find-files <file1>, ...} & Searched the database for packages that contain the given files and outputs their names, versions and architectures. The filenames are regular expressions as supported by the OCaml Str module. \\
		
		\texttt{-{}-get-dependencies <name1>, ...} & Print the direct dependencies of the specified packages. The package names are regular expressions as supported by OCaml's Str module. \\
		
		\texttt{-{}-get-reverse-dependencies} & Same as -{}-get-dependencies but retrieves the immediate reverse dependencies of a package.
	\end{tabularx}
	\egroup
	
	\subsection{General options}
	\bgroup
	\def\arraystretch{1.5}
	\begin{tabularx}{\textwidth}{lX}
		\texttt{-{}-db <file>} & Specifies the database file to use. The default is \file{pkgdb.xml} in the current working directory.
	\end{tabularx}
	\egroup
	
	\subsection{Options for queries}
	
	These options can be used to refine the results and the output of the query commands \texttt{-{}-find-files} and \texttt{-{}-get-dependencies}.
	
	\bgroup
	\def\arraystretch{1.5}
	\begin{tabularx}{\textwidth}{lX}
		\texttt{-{}-arch <arch>} & Filter the results by architecture \\
		
		\texttt{-{}-only-in-latest-version} & Do only search in the latest version of a package \\
		
		\texttt{-{}-print-only-names} & Do only print the package's names and no version or architecture or other information \\
	\end{tabularx}
	\egroup
	
	\section{File formats}
	\label{sec:file_formats}
	
	\subsection{\file{desc.xml}}
	\label{sec:desc.xml}
	
	\dirtree{.1 <sw file\_version=''1.1''> | <conf file\_version=''1.1''>.
		.2 <name>.
		.2 <version> \DTcomment{\{x.y.z\} with integer values x, y, z}.
		.2 <arch> \DTcomment{\{i386 | amd64\}}.
		.2 <file> \DTcomment{non-config file}.
		.2 <cfile sha512sum= > \DTcomment{config file}.
		.2 <dir> \DTcomment{directory}.
		.2 <rdep> \DTcomment{runtime dependency (package name)}.
	}

	\vspace{1em}
	The file, cfile, dir and rdep elements must be sorted ascending by their values according to OCaml's built in string comparison (uppercase letters are 'smaller' than lowercase letters)
	
	\subsection{\file{status.xml}}
	\label{sec:status.xml}
	
	This file is located in the \file{/var/lib/tpm} directory.
	
	\noindent
	\dirtree{.1 <status file\_version=''1.1''>.
		.2 <tuple> .
		.3 <sw \dots> | <conf \dots> \DTcomment{package element equal to one in \file{desc.xml}; The file\_version attribute must match the file\_version of status.}.
		.3 <reason> \DTcomment{\{auto | manua \}}.
		.3 <status> \DTcomment{\{installation | installed | removal | removed\}}.
	}

	\subsection{\file{pkgdb.xml}}
	\label{sec:pkgdb.xml}
	
	This file can be generated with \program{tpmdb}. It may have a different filename.
	
	\noindent
	\dirtree{.1 <pkgdb file\_version=''1.1''>.
		.2 <sw \dots> | <conf \dots> \DTcomment{package element equal to one in \file{desc.xml}; The file\_version attribute must match the file\_version of pkgdb.}.
	}

	\subsection{\file{config.xml}}
	\label{sec:config.xml}
	
	This file is located in the \file{/etc/tpm} directory.
	
	\noindent
	\dirtree{.1 <tpm file\_version=''1.0''>.
		.2 <repo type= \{dir\}> \DTcomment{path to Directory Repository}.
		.2 <arch> \DTcomment{\{i386 | amd64\}}.
	}

	\section{Repository formats}
	\label{sec:repository_formats}
	
	\subsection{Directory Repository}
	\label{sec:directory_repository}
	
	Directory Repositories are directory trees of the following format:
	
	\vspace{1em}	
	\noindent
	\begin{tabular}{rl}
		\file{\dots/} & \file{<arch>/<package-X1\_Y1.tpm.tar} \\
		& \hspace{0.5cm} $\vdots$ \\
		& \file{<arch>/<package-Xn\_Yn.tpm.tar} \\
	\end{tabular}

	\section{Caching}
	\label{sec:caching}
	
	No local caching is done. If scratchspace is needed (e.g. to unpack the transport form of a package, \file{/tmp/tpm} is used. This temporary directory is created in \file{/tmp} of the tools system, not within the runtime system's root file system.
	
	\section{Escaping of file names}
	\label{sec:escaping_of_file_names}
	
	Currently, file names are not escaped before they are included into XML files. This can lead to problems including code injection if file names include XML syntax elements. However I do not like nor encourage the concept of escaping in files that are processed by machines but rather use binary files. But I don't have the time for that.
	
	\section{Installed packages on a system}
	\label{sec:installed_packages_on_a_system}
	
	Within a system, the package name must be unique. That is, a package may only be installed once. This implies that at each point in time, only one architectural instance in exactly one version of a package can be installed.
	
	Moreover each file on a system belongs to exactly one package. There may not be two packages which provide the same file.
	
	\section{About command line output}
	\label{sec:about_command_line_output}
	
	I try to print enough information to understand possible problems and from were they originate. Additionally I try to create good looking output. However I do not have the time to format the output well in any case, thus I try to achieve a good looking format during normal operation and accept an ugly one in case of problems.
	
	\section{The package info location}
	\label{sec:the_package_info_location}
	
	For each package a directory with its name is created in \file{/var/lib/tpm} during installation. \file{/var/lib/tpm} is called package info location and the specific package's sub directory is called this package's package info location. Information which is better stored in a file than a database shall be located there (i.e. \file{postrm.sh}).
	
	\section{Special files}
	\label{sec:special_files}
	
	Special files, which are symbolic links, character and block devices, named pipes, and sockets, are treated like non-config files. Tar handles these kinds of files rightly therefore this decision is only relevant to config files, because these are directly processed by \program{tpm}. \program{tpm} treats them like non-config files, even if they are in a location in which only config files exists, because it does not make sense to calculate a checksum for those.
\end{document}